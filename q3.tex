\documentclass[a4paper, 12pt]{article}
\usepackage[margin=1.25in]{geometry}
\usepackage{bookmark}

\usepackage{amsmath}
\usepackage{amssymb}
\usepackage{nccmath}
\allowdisplaybreaks

\usepackage{graphicx}
\usepackage{float}

\usepackage{multirow}
\usepackage{tabularx}
\newcolumntype{L}{>{\centering\arraybackslash}X}

\renewcommand{\baselinestretch}{1.15}
\setlength{\parindent}{0pt}

\title{CS7.301: Assignment 3: Question 3}
\author{Himanshu Singh}
\date{\today}

\begin{document}

\section*{Question 2}

\subsection*{Part A: Flight to Oklahoma}

Since the plane is built to be able to fly on one engine, the only way it can fail to complete a four-hour flight is if no engine were working, or equivalently both the engines failed in the given flight. This condition is both necessary and sufficient.

\begin{equation}
P(\text{plane fails to complete the flight}) = P(\text{engine 1 fails} \cap \text{engine 2 fails})
\end{equation}

Since the two engines operate independently, their failures (an event in their operation cycle) too are independent of each other. Thus, we get

\begin{equation}
P(\text{plane fails to complete the flight}) = P(\text{engine 1 fails}) \cdot P(\text{engine 2 fails})
\end{equation}

Substituting $P(\text{engine 1 fails}) = P(\text{engine 2 fails}) = \dfrac{1}{100}$, we get

\begin{equation}
P(\text{plane fails to complete the flight}) = \dfrac{1}{10000} = 0.01\%
\end{equation}

\subsection*{Part B: Birthday Paradox}

\begin{fleqn}[\parindent]
\begin{equation}
\begin{split}
& P(\text{atleast two people have same birthday}) \\
&= 1 - P(\text{no two people have the same birthday}) \\
&= 1 - P(\text{all 30 people have distinct birthday}) \\
\end{split}
\end{equation}
\end{fleqn}

Since the birthdays are uniformly distributed over 365 days,

\begin{fleqn}[\parindent]
\begin{equation}
\begin{split}
&= 1 - \frac{\text{\#permutations of distinct 30 birthdays}}{\text{\#permutations of 30 birthdays}} \\
&= 1 - \frac{\cfrac{365!}{(365-30)!}}{365^{30}} \\
&\approx 1 - 0.294 \\
&= 0.706
\end{split}
\end{equation}
\end{fleqn}

\end{document}
