\documentclass[a4paper, 12pt]{article}
\usepackage[margin=1.25in]{geometry}
\usepackage{bookmark}

\usepackage{mathtools}
\usepackage{amsmath}
\usepackage{amssymb}
\usepackage{amsthm}
\allowdisplaybreaks
\newcommand\numberthis{\addtocounter{equation}{1}\tag{\theequation}}

\usepackage{graphicx}
\usepackage{float}

\usepackage{multirow}
\usepackage{tabularx}
\newcolumntype{L}{>{\centering\arraybackslash}X}

\renewcommand{\baselinestretch}{1.15}
\setlength{\parindent}{0pt}

\title{CS7.301: Assignment 3: Question 2}
\author{Himanshu Singh}
\date{\today}

\begin{document}

\section*{Question 2}

Without loss of generality, we assume that the $n$ independent tosses are numbered 1 to $n$. Consider the outcome of first toss, denoted by $X_1$. If the first toss comes up heads, the total number of heads is even iff the number of heads in the remaining $n - 1$ tosses is odd. Likewise, if the first toss comes up tails, the total number of heads is even iff the number of heads in the remaining $n - 1$ tosses is even. Applying the law of total probability, we can write
\begin{align*}
P(\text{even \#heads in n tosses}) &= P(\text{odd \#heads in rem. n - 1 tosses} \mid{} X_1 = H) \cdot P(X_1 = H) \\
& + P(\text{even \#heads in rem. n - 1 tosses} \mid{} X_1 = T) \cdot P(X_1 = T)
\intertext{Using the independence of remaining $n - 1$ tosses from the first toss, we get}
P(\text{even \#heads in n tosses}) &= P(\text{odd \#heads in rem. n - 1 tosses}) \cdot P(X_1 = H) \\
& + P(\text{even \#heads in rem. n - 1 tosses}) \cdot P(X_1 = T)
\end{align*}

Rewriting this in terms of the given notation, we get
\begin{align*}
q_n &= (1 - q_{n-1}) p + q_{n-1}(1 - p) \\
\intertext{Applying some simplifications, we arrive at the recurrence relation}
&= (1 - 2p) q_{n-1} + p \\
&= (1 - 2p) q_{n-1} + (-2p) \left(\frac{-1}{2}\right) - \frac{1}{2} + \frac{1}{2} \\
&= (1 - 2p) q_{n-1} + (1 - 2p) \left(\frac{-1}{2}\right) + \frac{1}{2} \\
q_n - \frac{1}{2} &= (1 - 2p) \left(q_{n-1} - \frac{1}{2}\right) \numberthis \label{eqn-rec} \\
\intertext{We will show by induction that}
q_n - \frac{1}{2} &= (1 - 2p)^i \left(q_{n-i} - \frac{1}{2}\right) \numberthis \label{eqn-ind} \\
\intertext{We know from \eqref{eqn-rec}, this holds for $i = 1$. Assuming it holds for some $i > 1$, we will show that it holds for $i + 1$. By assumption,}
q_n - \frac{1}{2} &= (1 - 2p)^i \left(q_{n-i} - \frac{1}{2}\right) \\
\intertext{Using \eqref{eqn-rec} for $(q_{n-i} - \frac{1}{2})$, we get}
&= (1 - 2p)^{i+1} \left(q_{n-i-1} - \frac{1}{2}\right) \\
\intertext{This resembles the exact form as \eqref{eqn-ind} for $i + 1$, completing the proof by induction. Substituting $i = n$ in \eqref{eqn-ind}, we get}
q_n - \frac{1}{2} &= (1 - 2p)^n \left(q_0 - \frac{1}{2}\right) \\
\intertext{It is trivial to see that the number of heads when no coin is tossed is zero, which is even. Thus, $q_0 = 1$.}
q_n &= \frac{1}{2} + (1 - 2p)^n \left(1 - \frac{1}{2}\right) \\
q_n &= \frac{1 + (1 - 2p)^n}{2}
\end{align*}

\qed

\end{document}
